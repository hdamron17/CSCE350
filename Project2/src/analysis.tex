\documentclass{article}
\usepackage{graphicx}
\usepackage{epstopdf}
\usepackage{subfig}
\usepackage[margin=1in]{geometry}

\graphicspath{./build}
\pagenumbering{gobble}

\title{CSCE 350 Project 2 Sorting Algorithm Analysis}
\author{Hunter Damron}
\date{17 Oct. 2018}

\begin{document}
  \maketitle

  \begin{figure}[h]
    \centering
    \subfloat[Number of comparisons for selection sort, quicksort, and the STL sort]{\resizebox{0.45\columnwidth}{!}{\input{build/plotgrowth3}}}
    \qquad
    \subfloat[Number of comparisons for only quicksort and the STL sort]{\resizebox{0.45\columnwidth}{!}{\input{build/plotgrowth2}}}
    \caption{Counts of comparisons in each of three sorting algorithms: selection sort, quicksort, and STL sort.}
    \label{fig:plots}
  \end{figure}

  Looking at the plots in Figure~\ref{fig:plots}, selection sort does appear to perform roughly $\frac{n^2}{2}$ comparisons which is consistent with the $\Theta(n^2)$ proven in class.
  In the plots, quicksort and STL sort both appear linear in the small viewpoint of the function.
  However, based on previous experience, I know that both of these should be in $\Theta(n \log n)$ which is also consistent with the plots since the value of $\log n$ only varies from $\log 200 \approx 2.3$ to $\log 6000 \approx 3.8$ which is not significant enough to be visible.
  If larger lists were sorted, this result would be much more noticeable.
  Because the quicksort appears to have the lowest order of growth, it would be the best for real use.
\end{document}
